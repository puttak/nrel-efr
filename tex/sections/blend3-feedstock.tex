% !TEX root = ../main.tex

\section{Blend3 feedstock}

This section provides information about the ``Blend3'' feedstock as used for the FCIC project. See the sections below for general information, chemical composition, particle characterization, and reactor experiments associated with this feedstock.

\subsection{Information}

\begin{table}[H]
    \centering
    \caption{General information for the Blend3 feedstock.}
    \begin{tabular}{ll}
        \toprule
        Item    & Description \\
        \midrule
        Name    & Blend3 \\
        ID      & ? \\
        Contact & ? \\
        \bottomrule
    \end{tabular}
\end{table}

\subsection{Proximate and ultimate analyses}

\begin{table}[H]
    \centering
    \caption{Blend3 proximate analysis mass percent, as-received basis. Source \cite{Choratch-2017}.}
    \begin{tabular}{lrrr}
        \toprule
        Proximate & \% ar & \% ar & \% ar \\
        \midrule
        FC        & 16.92 & ? & ? \\
        VM        & 76.40 & ? & ? \\
        ash       & 0.64  & ? & ? \\
        moisture  & 6.04  & ? & ? \\
        \bottomrule
    \end{tabular}
\end{table}

\begin{table}[H]
    \centering
    \caption{Blend3 ultimate analysis mass percent, as-received basis. Source \cite{Choratch-2017}.}
    \begin{tabular}{lrrr}
        \toprule
        Element & \% ar & \% ar & \% ar \\
        \midrule
        C        & 49.52   & ? & ? \\
        H        & 5.28    & ? & ? \\
        O        & 38.35   & ? & ? \\
        N        & 0.15    & ? & ? \\
        S        & 0.02    & ? & ? \\
        ash      & 0.64    & ? & ? \\
        moisture & 6.04    & ? & ? \\
        \bottomrule
    \end{tabular}
\end{table}

\subsection{Chemical analysis}

The chemical components listed below are grouped into cellulose, hemicellulose, lignin, tann, tgl, and ash for use in the Debiagi 2018 biomass pyrolysis kinetics \cite{Debiagi-2018}. Celluluse is represented by glucan. Hemicellulose is acetyl, arabinan, galactan, mannan, and xylan. Tannins (TANN) are represented by free fructose, free glucose, sucrose, and water extractives while triglycerides (TGL) are ethanol extractives. Ash is the non-structural and structural inorganics.

\begin{table}[H]
    \centering
    \caption{Blend3 chemical composition analysis mass percent, dry basis. Components are grouped into cellulose, hemicellulose, lignin, tann, and tgl for use in the Debiagi pyrolysis kinetics. Source \cite{Starace-2020}.}
    \label{tab:chem-components}
    \begin{tabular}{llrrr}
        \toprule
        Chemical component & Group & \% dry & \% dry & \% dry \\
        \midrule
        glucan                    & cellulose      & 38.95 & ? & ? \\
        acetyl                    & hemicellulose  & 1.59  & ? & ? \\
        arabinan                  & hemicellulose  & 1.40  & ? & ? \\
        galactan                  & hemicellulose  & 3.16  & ? & ? \\
        mannan                    & hemicellulose  & 10.52 & ? & ? \\
        xylan                     & hemicellulose  & 7.89  & ? & ? \\
        lignin                    & lignin         & 29.48 & ? & ? \\
        free fructose             & tann           & 0.07  & ? & ? \\
        free glucose              & tann           & 0.04  & ? & ? \\
        sucrose                   & tann           & 0.04  & ? & ? \\
        water extractives         & tann           & 2.75  & ? & ? \\
        ethanol extractives       & tgl            & 3.49  & ? & ? \\
        non-structural inorganics & ash            & 0.22  & ? & ? \\
        structural inorganics     & ash            & 0.41  & ? & ? \\
        \bottomrule
    \end{tabular}
\end{table}

\begin{table}[H]
    \centering
    \caption{Blend3 biomass composition mass percent, dry basis. Values are calculated from Table \ref{tab:chem-components}.}
    \begin{tabular}{lr}
        \toprule
        Biomass composition & \% dry \\
        \midrule
        cellulose     & 38.95 \\
        hemicellulose & 24.56 \\
        lignin        & 29.48 \\
        tann          & 2.90  \\
        tgl           & 3.49  \\
        ash           & 0.63  \\
        \bottomrule
    \end{tabular}
\end{table}

\begin{table}[H]
    \centering
    \caption{Blend3 ash analysis as weight percent of ash. Source \cite{Choratch-2017}.}
    \begin{tabular}{lrrr}
        \toprule
        Metal oxide & wt. \% & wt. \% & wt. \% \\
        \midrule
        SiO$_2$     & 28.1 & ? & ? \\
        Al$_2$O$_3$ & 7.06 & ? & ? \\
        TiO$_2$     & 0.34 & ? & ? \\
        CaO         & 21.8 & ? & ? \\
        Na$_2$O     & 0.71 & ? & ? \\
        K$_2$O      & 13.8 & ? & ? \\
        P$_2$O$_5$  & 5.47 & ? & ? \\
        SO$_3$      & 1.23 & ? & ? \\
        Cl          & 0.09 & ? & ? \\
        CO$_2$      & 5.14 & ? & ? \\
        \bottomrule
    \end{tabular}
\end{table}

\subsection{Particle characterization}

\begin{table}[H]
    \centering
    \caption{Blend3 particle properties from pelletized crushed feedstock. The crushed feedstock is used in the entrained flow reactor.}
    \begin{tabular}{crlc}
        \toprule
        Property & Value & Description & Source \\
        \midrule
        $\rho$  & 1,050 kg/m$^3$ & particle density, daf basis & \cite{Pecha-2018} \\
        $\eta$  & 0.27           & particle porosity & \\
        $k$     & 0.23 W/mK      & thermal conductivity & \\
        \bottomrule
    \end{tabular}
\end{table}

\subsection{Reactor yields}

\begin{table}[H]
    \centering
    \caption{Entrained flow reactor yields for Blend3 feedstock.}
    \begin{tabular}{lr}
        \toprule
        Yield & wt. \% \\
        \midrule
        total liquid   & 64.9 \\
        char           & 13.9 $\pm$ 0.1 \\
        gas            & 17.2 $\pm$ 0.2 \\
        mass balance   & 96.9 $\pm$ 1.5 \\
        carbon balance & 93.0 $\pm$ 1.0 \\
        \bottomrule
    \end{tabular}
\end{table}
