% !TEX root = ../main.tex

\section{Model development}

Details about the biomass pyrolysis kinetics, biomass characterization method, and computational models developed for the entrained flow reactor are discussed in the following sections.

\subsection{Pyrolysis kinetics}

The kinetic reaction mechanisms presented in the Debiagi et al. 2018 paper are used to model biomass pyrolysis in the entrained flow reactor \cite{Debiagi-2018}. Table \ref{tab:chem-kinetics} summarizes the reactions along with the associated prefactors and activation energies. A description of the chemical species in the Debiagi et al. kinetic scheme is provided in Table \ref{tab:chem-species}. Species are grouped into solid, metaplastic, gas, and liquid phases.

\begin{center}
    \footnotesize
    \setlength\LTleft{-1in}
    \setlength\LTright{-1in}
    \begin{longtable}{cp{4in}lr}
        \caption{Kinetic reactions for biomass pyrolysis where A is the prefactor, E is the activation energy, and T is temperature. Source \cite{Debiagi-2018}.}
        \label{tab:chem-kinetics} \\
        \toprule
        Item & Reaction & A (1/s) & E (cal/mol) \\
        \midrule
        1  & CELL $\rightarrow$ CELLA & 1.5 $\times$ 10$^{14}$ & 47,000 \\
        2  & CELLA $\rightarrow$ 0.40 CH2OHCHO + 0.03 CHOCHO + 0.17 CH3CHO + 0.25 C6H6O3 + 0.35 C2H5CHO + 0.20 CH3OH + 0.15 CH2O + 0.49 CO + 0.05 G\{CO\} + 0.43 CO2 + 0.13 H2 + 0.93 H2O + 0.05 G\{COH2\} loose + 0.02 HCOOH + 0.05 CH2OHCH2CHO + 0.05 CH4 + 0.1 G\{H2\} + 0.66 CHAR & 2.5 $\times$ 10$^6$ & 19,100 \\
        3  & CELLA $\rightarrow$ C6H10O5 & 3.3 $\times$ T & 10,000 \\
        4  & CELL $\rightarrow$ 4.45 H2O + 5.45 CHAR + 0.12 G\{COH2\} stiff + 0.18 G\{COH2\} loose + 0.25 G\{CO\} + 0.125 G\{H2\} + 0.125 H2 & 9.0 $\times$ 10$^7$ & 31,000 \\
        5  & GMSW $\rightarrow$ 0.70 HCE1 + 0.30 HCE2 & 1.0 $\times$ 10$^{10}$ & 31,000 \\
        6  & XYHW $\rightarrow$ 0.35 HCE1 + 0.65 HCE2 & 1.25 $\times$ 10$^{11}$ & 31,400 \\
        7  & XYGR $\rightarrow$ 0.12 HCE1 + 0.88 HCE2 & 1.25 $\times$ 10$^{11}$ & 30,000 \\
        8  & HCE1 $\rightarrow$ 0.25 C5H8O4 + 0.25 C6H10O5 + 0.16 FURFURAL + 0.13 C6H6O3 + 0.09 CO2 + 0.1 CH4 + 0.54 H2O + 0.06 CH2OHCH2CHO + 0.1 CHOCHO + 0.02 H2 + 0.1 CHAR & 16.0 $\times$ T & 12,900 \\
        9  & HCE1 $\rightarrow$ 0.4 H2O + 0.39 CO2 + 0.05 HCOOH + 0.49 CO + 0.01 G\{CO\} + 0.51 G\{CO2\} + 0.05 G\{H2\} + 0.4 CH2O + 0.43 G\{COH2\} loose + 0.3 CH4 + 0.325 G\{CH4\} + 0.1 C2H4 + 0.075 G\{C2H4\} + 0.975 CHAR + 0.37 G\{COH2\} stiff + 0.1 H2 + 0.2 G\{C2H6\} & 3.0 $\times$ 10$^{-3}$ $\times$ T & 3,600 \\
        10 & HCE2 $\rightarrow$ 0.3 CO + 0.5125 CO2 + 0.1895 CH4 + 0.5505 H2 + 0.056 H2O + 0.049 C2H5OH + 0.035 CH2OHCHO + 0.105 CH3CO2H + 0.0175 HCOOH + 0.145 FURFURAL + 0.05 G\{CH4\} + 0.105 G\{CH3OH\} + 0.1 G\{C2H4\} + 0.45 G\{CO2\} + 0.18 G\{COH2\} loose + 0.7125 CHAR + 0.21 G\{H2\} + 0.78 G\{COH2\} stiff + 0.2 G\{C2H6\} & 7.0 $\times$ 10$^9$ & 30,500 \\
        11 & LIGH $\rightarrow$ LIGOH + 0.5 C2H5CHO + 0.4 C2H4 + 0.2 CH2OHCHO + 0.1 CO + 0.1 C2H6 & 6.7 $\times$ 10$^{12}$ & 37,500 \\
        12 & LIGO $\rightarrow$ LIGOH + CO2 & 3.3 $\times$ 10$^8$ & 25,500 \\
        13 & LIGC $\rightarrow$ 0.35 LIGCC + 0.1 VANILLIN + 0.1 C6H5OCH3 + 0.27 C2H4 + H2O + 0.17 G\{COH2\} loose + 0.4 G\{COH2\} stiff + 0.22 CH2O + 0.21 CO + 0.1 CO2 + 0.36 G\{CH4\} + 5.85 CHAR + 0.2 G\{C2H6\} + 0.1 G\{H2\} & 1.0 $\times$ 10$^{11}$ & 37,200 \\
        14 & LIGCC $\rightarrow$ 0.25 VANILLIN + 0.15 CRESOL + 0.15 C6H5OCH3 + 0.35 CH2OHCHO + 0.7 H2O + 0.45 CH4 + 0.3 C2H4 + 0.7 H2 + 1.15 CO + 0.4 G\{CO\} + 6.80 CHAR + 0.4 C2H6 & 1.0 $\times$ 10$^4$ & 24,800 \\
        15 & LIGOH $\rightarrow$ 0.9 LIG + H2O + 0.1 CH4 + 0.6 CH3OH + 0.3 G\{CH3OH\} + 0.05 CO2 + 0.65 CO + 0.6 G\{CO\} + 0.05 HCOOH + 0.45 G\{COH2\} loose + 0.4 G\{COH2\} stiff + 0.25 G\{CH4\} + 0.1 G\{C2H4\} + 0.15 G\{C2H6\} + 4.25 CHAR + 0.025 C24H28O4 + 0.1 C2H3CHO & 1.5 $\times$ 10$^8$ & 30,000 \\
        16 & LIG $\rightarrow$ VANILLIN + 0.1 C6H5OCH3 + 0.5 C2H4 + 0.6 CO + 0.3 CH3CHO + 0.1 CHAR & 4.0 $\times$ T & 12,000 \\
        17 & LIG $\rightarrow$ 0.6 H2O + 0.3 CO + 0.1 CO2 + 0.2 CH4 + 0.4 CH2O + 0.2 G\{CO\} + 0.4 G\{CH4\} + 0.5 G\{C2H4\} + 0.4 G\{CH3OH\} + 1.25 G\{COH2\} loose + 0.65 G\{COH2\} stiff + 6.1 CHAR + 0.1 G\{H2\} & 8.3 $\times$ 10$^{-2}$ $\times$ T & 8,000 \\
        18 & LIG $\rightarrow$ 0.6 H2O + 2.6 CO + 0.6 CH4 + 0.4 CH2O + 0.75 C2H4 + 0.4 CH3OH + 4.5 CHAR + 0.5 C2H6 & 1.5 $\times$ 10$^9$ & 31,500 \\
        19 & TGL $\rightarrow$ C2H3CHO + 2.5 MLINO + 0.5 U2ME12 & 7.0 $\times$ 10$^{12}$ & 45,700 \\
        20 & TANN $\rightarrow$ 0.85 C6H5OH + 0.15 G\{C6H5OH\} + G\{CO\} + H2O + ITANN & 2.0 $\times$ 10$^1$ & 10,000 \\
        21 & ITANN $\rightarrow$ 5 CHAR + 2 CO + H2O + 0.55 G\{COH2\} loose + 0.45 G\{COH2\} stiff & 1.0 $\times$ 10$^3$ & 25,000 \\
        22 & G\{CO2\} $\rightarrow$ CO2 & 1.0 $\times$ 10$^6$ & 24,500 \\
        23 & G\{CO\} $\rightarrow$ CO & 5.0 $\times$ 10$^{12}$ & 52,500 \\
        24 & G\{CH3OH\} $\rightarrow$ CH3OH & 2.0 $\times$ 10$^{12}$ & 50,000 \\
        25 & G\{COH2\}loose $\rightarrow$ 0.2 CO + 0.2 H2 + 0.8 H2O + 0.8 CHAR & 6.0 $\times$ 10$^{10}$ & 50,000 \\
        26 & G\{C2H6\} $\rightarrow$ C2H6 & 1.0 $\times$ 10$^{11}$ & 52,000 \\
        27 & G\{CH4\} $\rightarrow$ CH4 & 1.0 $\times$ 10$^{11}$ & 53,000 \\
        28 & G\{C2H4\} $\rightarrow$ C2H4 & 1.0 $\times$ 10$^{11}$ & 54,000 \\
        29 & G\{C6H5OH\} $\rightarrow$ C6H5OH & 1.5 $\times$ 10$^{12}$ & 55,000 \\
        30 & G\{COH2\}stiff $\rightarrow$ 0.8 CO + 0.8 H2 + 0.2 H2O + 0.2 CHAR & 1.0 $\times$ 10$^9$ & 59,000 \\
        31 & G\{H2\} $\rightarrow$ H2 & 1.0 $\times$ 10$^8$ & 70,000 \\
        32 & ACQUA $\rightarrow$ H2O & 1.0 $\times$ T & 8,000 \\
        \bottomrule
    \end{longtable}
\end{center}

\begin{center}
    \footnotesize
    \begin{longtable}{cllll}
        \caption{Description of the chemical species in the Debiagi kinetics scheme for biomass pyrolysis. Source \cite{Debiagi-2018}.}
        \label{tab:chem-species} \\
        \toprule
        Item & Name & Formula & Phase & Description \\
        \midrule
        1  & CELL           & C$_6$H$_{10}$O$_5$      & \cellcolor{green!25}solid        & cellulose \\
        2  & CELLA          & C$_6$H$_{10}$O$_5$      & \cellcolor{green!25}solid        & active cellulose \\
        3  & GMSW           & C$_5$H$_{8}$O$_4$       & \cellcolor{green!25}solid        & hemicellulose softwood \\
        4  & XYHW           & C$_5$H$_{8}$O$_4$       & \cellcolor{green!25}solid        & hemicellulose hardwood \\
        5  & XYGR           & C$_5$H$_{8}$O$_4$       & \cellcolor{green!25}solid        & hemicellulose grass \\
        6  & HCE1           & C$_5$H$_{8}$O$_4$       & \cellcolor{green!25}solid        & intermediate hemicellulose \\
        7  & HCE2           & C$_5$H$_{8}$O$_4$       & \cellcolor{green!25}solid        & intermediate hemicellulose \\
        8  & ITANN          & C$_8$H$_{4}$O$_4$       & \cellcolor{green!25}solid        & intermediate phenolics \\
        9  & LIG            & C$_{11}$H$_{12}$O$_4$   & \cellcolor{green!25}solid        & intermediate lignin \\
        10 & LIGC           & C$_{15}$H$_{14}$O$_4$   & \cellcolor{green!25}solid        & carbon rich lignin \\
        11 & LIGCC          & C$_{15}$H$_{14}$O$_4$   & \cellcolor{green!25}solid        & intermediate lignin \\
        12 & LIGH           & C$_{22}$H$_{28}$O$_9$   & \cellcolor{green!25}solid        & hydrogen rich lignin \\
        13 & LIGO           & C$_{20}$H$_{22}$O$_{10}$& \cellcolor{green!25}solid        & oxygen rich lignin \\
        14 & LIGOH          & C$_{19}$H$_{22}$O$_8$   & \cellcolor{green!25}solid        & intermediate lignin \\
        15 & TANN           & C$_{15}$H$_{12}$O$_7$   & \cellcolor{green!25}solid        & tannins \\
        16 & TGL            & C$_{57}$H$_{100}$O$_7$  & \cellcolor{green!25}solid        & triglycerides \\
        17 & CHAR           & C                       & \cellcolor{green!25}solid        & char as pure carbon \\
        18 & G\{COH2\} loose& CH$_2$O                 & \cellcolor{orange!25}metaplastic & loose formaldehyde \\
        19 & G\{CO2\}       & CO$_2$                  & \cellcolor{orange!25}metaplastic & trapped carbon dioxide \\
        20 & G\{CO\}        & CO                      & \cellcolor{orange!25}metaplastic & trapped carbon monoxide \\
        21 & G\{CH3OH\}     & CH$_4$O                 & \cellcolor{orange!25}metaplastic & trapped methanol \\
        22 & G\{CH4\}       & CH$_4$                  & \cellcolor{orange!25}metaplastic & trapped methane \\
        23 & G\{C2H4\}      & C$_2$H$_4$              & \cellcolor{orange!25}metaplastic & trapped ethylene \\
        24 & G\{C6H5OH\}    & C$_6$H$_6$O             & \cellcolor{orange!25}metaplastic & trapped phenol \\
        25 & G\{COH2\} stiff& CH$_2$O                 & \cellcolor{orange!25}metaplastic & stiff formaldehyde \\
        26 & G\{H2\}        & H$_2$                   & \cellcolor{orange!25}metaplastic & trapped hydrogen \\
        27 & G\{C2H6\}      & C$_2$H$_6$              & \cellcolor{orange!25}metaplastic & trapped ethane \\
        28 & C2H4           & C$_2$H$_4$              & \cellcolor{purple!25}gas         & ethylene \\
        29 & C2H6           & C$_2$H$_6$              & \cellcolor{purple!25}gas         & ethane \\
        30 & CH2O           & CH$_2$O                 & \cellcolor{purple!25}gas         & formaldehyde \\
        31 & CH4            & CH$_4$                  & \cellcolor{purple!25}gas         & methane \\
        32 & CO             & CO                      & \cellcolor{purple!25}gas         & carbon monoxide \\
        33 & CO2            & CO$_2$                  & \cellcolor{purple!25}gas         & carbon dioxide \\
        34 & H2             & H$_2$                   & \cellcolor{purple!25}gas         & hydrogen \\
        35 & C2H3CHO        & C$_3$H$_4$O             & \cellcolor{blue!25}liquid        & acrolein \\
        36 & C2H5CHO        & C$_3$H$_6$O             & \cellcolor{blue!25}liquid        & propionaldehyde \\
        37 & C2H5OH         & C$_2$H$_6$O             & \cellcolor{blue!25}liquid        & ethanol \\
        38 & C5H8O4         & C$_5$H$_8$O$_4$         & \cellcolor{blue!25}liquid        & xylofuranose \\
        39 & C6H10O5        & C$_6$H$_{10}$O$_5$      & \cellcolor{blue!25}liquid        & levoglucosan \\
        40 & C6H5OCH3       & C$_7$H$_8$O             & \cellcolor{blue!25}liquid        & anisole \\
        41 & C6H5OH         & C$_6$H$_6$O             & \cellcolor{blue!25}liquid        & phenol \\
        42 & C6H6O3         & C$_6$H$_6$O$_3$         & \cellcolor{blue!25}liquid        & hydroxymethylfurfural \\
        43 & C24H28O4       & C$_{24}$H$_{28}$O$_4$   & \cellcolor{blue!25}liquid        & heavy molecular weight lignin \\
        44 & CH2OHCH2CHO    & C$_3$H$_6$O$_2$         & \cellcolor{blue!25}liquid        & propionic acid \\
        45 & CH2OHCHO       & C$_2$H$_4$O$_2$         & \cellcolor{blue!25}liquid        & acetic acid \\
        46 & CH3CHO         & C$_2$H$_4$O             & \cellcolor{blue!25}liquid        & acetaldehyde \\
        47 & CH3CO2H        & C$_2$H$_4$O$_2$         & \cellcolor{blue!25}liquid        & acetic acid \\
        48 & CH3OH          & CH$_4$O                 & \cellcolor{blue!25}liquid        & methanol \\
        49 & CHOCHO         & C$_2$H$_2$O$_2$         & \cellcolor{blue!25}liquid        & glyoxal \\
        50 & CRESOL         & C$_7$H$_8$O             & \cellcolor{blue!25}liquid        & cresol \\
        51 & FURFURAL       & C$_5$H$_4$O$_2$         & \cellcolor{blue!25}liquid        & 2-furaldehyde \\
        52 & H2O            & H$_2$O                  & \cellcolor{blue!25}liquid        & water from reactions \\
        53 & HCOOH          & CH$_2$O$_2$             & \cellcolor{blue!25}liquid        & formic acid \\
        54 & MLINO          & C$_{19}$H$_{34}$O$_2$   & \cellcolor{blue!25}liquid        & methyl linoleate \\
        55 & U2ME12         & C$_{13}$H$_{22}$O$_2$   & \cellcolor{blue!25}liquid        & linalyl propionate \\
        56 & VANILLIN       & C$_8$H$_8$O$_3$         & \cellcolor{blue!25}liquid        & vanillin \\
        57 & ACQUA          & H$_2$O                  & \cellcolor{blue!25}liquid        & water within biomass \\
        \bottomrule
    \end{longtable}
\end{center}

\subsection{Biomass characterization}

The Debiagi kinetics rely on an initial biomass composition defined as cellulolose, hemicellulose, lignin-c, lignin-h, lignin-o, tannins, and triglycerides. According to the Debiagi et al. 2015 paper, the chemical components of the biomass are defined as shown below in Table \ref{tab:chem-components} \cite{Debiagi-2015}. The Debiagi article does not provide information on how to experimentally determine these components; therefore, the reader must decide on appropriate measurement techniques.

\begin{table}[H]
    \centering
    \caption{Chemical components of biomass according to Debiagi et al. \cite{Debiagi-2015}.}
    \label{tab:chem-components}
    \begin{tabular}{lp{2.2in}}
        \toprule
        Biomass composition & Description \\
        \midrule
        cellulose     & glucan \\
        \addlinespace[0.1in]
        hemicellulose & mixture of sugars such as hexoses and pentoses; mainly xylose, mannose, galactose, and arabinose \\
        \addlinespace[0.1in]
        lignin        & aromatic alcohols such as coniferyl, sinapyl, p-coumaryl alcohol \\
        \addlinespace[0.1in]
        lignin-c      & carbon-rich lignin \\
        \addlinespace[0.1in]
        lignin-h      & hydrogen-rich lignin \\
        \addlinespace[0.1in]
        lignin-o      & oxygen-rich lignin \\
        \addlinespace[0.1in]
        tannins       & hydrophilic extractives, phenolics, ethanol and water, represented by a gallocatechin polymer \\
        \addlinespace[0.1in]
        triglycerides & hydrophobic extractives, hexane and ether, linoleic acid \\
        \bottomrule
    \end{tabular}
\end{table}

Ideally, the composition of the biomass would be directly measured; otherwise, the characterization method discussed in the Debiagi paper estimates the composition based on elemental analysis data \cite{Debiagi-2015}. The characterization method utilizes carbon and hydrogen obtained from elemental (ultimate) analysis of the biomass to predict the biochemical composition in terms of cellulose, hemicellulose, and lignin. Splitting parameters $\alpha$, $\beta$, $\gamma$, $\delta$, $\epsilon$ are used to improve the validity of the characterization procedure by accounting for extractives in the biomass.

\subsection{Batch reactor}

The material balance for a typical chemical reactor is shown in Equation \ref{eq:typical-balance} where $C_0$ is inlet concentration, $C$ is outlet concentration, $v$ is volumetric flow rate, $r$ is the reaction rate, and $V$ is the reactor volume.

\begin{equation}
    \label{eq:typical-balance}
    \begin{aligned}
        accumulation &= input - output + reaction \\
        \frac{dC}{dt} V &= v C_0 - v C + r V
    \end{aligned}
\end{equation}

A batch reactor was modeled to understand the time scales associated with the biomass pyrolysis kinetics. For the batch reactor, input and output is zero therefore only the accumulation and reaction terms remain in the material balance. For a constant volumne reactor the $V$ terms cancel out; therefore, Equation \ref{eq:batch-balance} represents the material balance for a batch reactor model.

\begin{equation}
    \label{eq:batch-balance}
    \begin{aligned}
        accumulation &= 0 - 0 + reaction \\
        \frac{dC}{dt} &= r
    \end{aligned}
\end{equation}

\subsection{Sensitivity analysis}

A sensitivity analysis was performed with the Debiagi pyrolysis kinetics to investigate the effects of biomass composition on product yields. The awesome SALib Python package was utilized for sample generation and prediction of the Sobol indices \cite{Herman-2017}. For the sensitivity analysis model, a sample represents the biomass composition as cellulose, hemicellulose, lignin-c, lignin-h, lignin-o, tannins, and triglycerides. This sample (or composition) is used in a reactor model at a certain temperature and pressure to predict pyrolysis yields. The sensitivity analysis model applies this approach to a large sample matrix then uses the generated data to perform a Sobol analysis.
